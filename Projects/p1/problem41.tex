%
% do not add anything in this part
%
\ifx\onefile\undefined

	\documentclass{article}

	%if tcolorbox and tikz are installed use next line
	%\usepackage[tcbox]{projectalgo}
	\usepackage{projectalgo}

	% replace type by one of graph, math, combinatory, string, network, datastructure, ai, image
	\pbtype{type}

	\begin{document}
\fi

%
% things can be added below
%

\def\pbname{Karatsuba's multiplication} %change this, do not use any number, just the name

\section{\pbname} 

% only for overview, so short description (no more than 1-2 lines)
\begin{overview}
\item [Algorithm:]Karatsuba's multiplication~(algo.~\ref{problem41}) 
	% - replace nb with problem number (e.g. problem101)
	% -	must match the label of the algorithm 
	% - for more than one algo list each of them and use problem101a, problem101b, problem101c etc.
\item [Input:] Two integers $a,b$, number of digits in an integer $n$
\item [Complexity:] $\mathcal{O}(n^{log_23})$
\item [Data structure compatibility:] N/A
\item [Common applications:] Large number multiplication
\end{overview}



\begin{problem}{\pbname}
	Give the precise product of two integers which might have many digits
\end{problem}

\subsection*{Description}

In practice there are many scenarios when multiplication of two large numbers is needed. Since the original multiplication function implemented in computer might be limited with regard to precision, algorithms wit hhigh precision to perform large number operation is needed.

One intuitive way is to perform multiplication like the school kids do: multiply every digit in one number with every digit from another, adding proper number of zeros after the product and then sum all results. It is a correct result, but when the number of digits is large enough, this algorithm turns out to be much too time consuming.

Karatsuba's multiplication uses the idea of divide and conquer to solve this problem. Instead of being divided into single digits, the two numbers will first be divided into two parts with same length. 

Then by properly perform addition/subtraction of the four parts, the number of recursive call of multiplication can be reduced from 4 to 3. Since multiplication's time cost is much more than that of addition, this will save a lot of time.

Because of the feature that the recursive call is much more time consuming, we can just focus on the time of multiplication. Using recurrance relationship it can be seen that the time of multiplying two n-digit numbers is 3 times that of two $\frac{n}{2}$-digit numbers. So the time of multiplying two n-digit numbers in Karatsuba's algorithm can be roughly presented as $$T(n) = T(1) \times 3^{log_2n}=T(1)\times n^{log_23}=O(n^{log_23})$$

{\bf Note: } for convenience, $n$ is set to be a power of $2$. If necessary, the two numbers can be prefixed with zeros.

% add comment in the pseudocode: \cmt{comment}
% define a function name: \SetKwFunction{shortname}{Name of the function}
% use the defined function: \shortname{$variables$}
% use the keyword ``function'': \Fn{function name}, e.g. \Fn{\shortname{$var$}}
\begin{Algorithm}[Karatsuba's multiplication\label{problem41}][H]
	% - replace nb with problem number (e.g. problem101)
	% -	must match the reference in the overview
	% - when writing more than one algo use problem101a, problem101b, problem101c etc.
	\SetKwFunction{Kara}{Karatsuba}	
	\Input{Integers $a,b$,number of digits $n$}
	\Output{Product of $a$ and $b$}
		\Fn{\Kara{$a,b,n$}}{
			\uIf{n=1} 
			{\Ret{$a\times b$}\;}
			sign $\leftarrow$ 1\;
			\uIf{$a<0$}{
				$a \leftarrow -a$\;
				$sign \leftarrow -sign$\;
			}
			\uIf{$b<0$}{
				$b \leftarrow -b$\;
				$sign \leftarrow -sign$\;
			}
			a1 $\leftarrow \lfloor \frac{a}{10^{\frac{n}{2}}}\rfloor $\;
			b1 $\leftarrow \lfloor \frac{b}{10^{\frac{n}{2}}}\rfloor $\;
			a2 $\leftarrow a-a1\times 10^{\frac{n}{2}}$\;
			b2 $\leftarrow b-b1\times 10^{\frac{n}{2}}$\;
			U $\leftarrow$ \Kara{$a1,b1,\frac{n}{2}$}\;
			V $\leftarrow$ \Kara{$a2,b2,\frac{n}{2}$}\;
			W $\leftarrow$ \Kara{$a1-a2,b1-b2,\frac{n}{2}$}\;
			Z $\leftarrow$ U+V-W\;
			\Ret{$(10^nU+10^{\frac{n}{2}}Z+V)\times sign$}\
		}
	\BlankLine

	\Ret{\Kara{$a,b,n$}}

\end{Algorithm}

\subsection*{References}
% list references where to find information on the given problem
% prefer books, research articles, or internet sources that are likely to remain available over time
% as much as possible offer several options, including at least one which provide a detailed study of the problem
% if available include links to programs/code solving the problem

\begin{itemize}\itemsep .125cm
	\item \url{http://people.cs.uchicago.edu/~laci/HANDOUTS/karatsuba.pdf}
	\item A. A. Karatsuba (1995). "The Complexity of Computations" (PDF). Proceedings of the Steklov Institute of Mathematics. 
	\item \url{https://en.wikipedia.org/wiki/Karatsuba_algorithm}
\end{itemize}

\ifx\onefile\undefined
	\end{document}
\fi
