\documentclass{beamer}
\usepackage{bookmark}
\usepackage{beamerthemeBerkeley}
\usepackage{color}
\usepackage{amsmath}
\usepackage{hyperref}
\hypersetup{colorlinks,linkcolor=blue,citecolor=blue}
\usepackage{listings}
\usepackage{xcolor}
\usepackage{graphicx}
\usepackage{amsfonts}
\usecolortheme{dove}
\lstset{
numbers=left,
showspaces=false,
showstringspaces=false,
xleftmargin=.05\textwidth,
frame=shadowbox,
commentstyle=\color{red!80!green!80!blue!80},
}
\title{Matrix Inversion}
\author{Group 23}
\institute{UM-SJTU Joint Institute}
\date{\today}

\begin{document}
\begin{frame}
	\titlepage
\end{frame}

\begin{frame}
	\tableofcontents
\end{frame}


\begin{frame}
	\section{Introduction}
	\frametitle{Matrix Inversion}
	\begin{block}{Definition}
		The {\bf  inverse} of a $n\times n$ matrix ${\bf A}$, denoted as ${\bf A^{-1}}$, is a matrix that satisfies the expression $${\bf AA^{-1}=A^{-1}A=I_n}$$

		Where ${\bf I_n}$ is a $n\times n$ matrix with $1$ on its diagonal and $0$ otherwise.
	\end{block}
\end{frame}

\begin{frame}
	\frametitle{Naive Matrix Inversion Method}
	The naive method to solve matrix inversion is to treat the expression ${\bf AA^{-1}=I_n}$ as a linear equation system whose unkonwns are $1\times n$ vectors (rows of ${\bf A^{-1}}$)

	\vskip 2em
	Solving this equation is of $\mathcal{O}(n^4)$ complexity (Gaussian Elimination)
\end{frame}

\begin{frame}
	\section{Algorithm}
	\frametitle{Two Matrix Inversion Algorithms}
	\begin{itemize}
		\item Cholesky Decomposition 
		\begin{itemize}
			\item Complexity: $\mathcal{O}(n^3)$
			\item Application: Positive Definite Hermitian matrices
		\end{itemize}
		\item Levinson-Durbin Recursion
		\begin{itemize}
			\item Complexity: $\mathcal{O}(n^3)$
			\item Application: Symmetric Toeplitz matrices
		\end{itemize}
	\end{itemize}
\end{frame}

\begin{frame}
	\subsection{Cholesky Decomposition}
	\frametitle{Cholesky Decomposition}
	A positive definite symmetric matrix ${\bf A}$ can be represented as product of two transpose triangular matrix as the following $${\bf A = U^TU}$$ where ${\bf U}$ is an $n\times n$ upper triangular matrix.

	{\bf Note: }The implementation of finding ${\bf U}$ for a certain ${\bf A}$ can be in-place, so that no extra space is needed.
	
\end{frame}

\begin{frame}
	Based on the Cholesky decomposition, the euqation can be transformed as $${\bf U^TUA^{-1}=I_n \Rightarrow \begin{cases}\bf U^TB &= \bf I_n\\\bf UA^{-1}&=\bf B\end{cases}}$$ and solving these two equations is simple because back/forward substitution can be directly applied. 
		
	What's more, the symmetric property of ${\bf A}$ will also cause a reduction in the calculation.
\end{frame}

\begin{frame}
	\subsection{Levinson-Durbin Recursion}
	\frametitle{Levinson-Durbin Recursion}
The first step of Levinson Durbin is to partition the matrix into blocks $$ {\bf A = \left[\begin{matrix}\bf B_{(n-1)\times(n-1)} & \bf C_{(n-1)\times 1} \\\bf D_{1\times (n-1)} & \bf E_{1\times 1}\end{matrix}\right]}$$

The algorithm will first find a inversion of ${\bf B}$, denoted as ${\bf B^{-1}}$ using recursive call and also find a {\bf backward vector} of ${\bf A}$, denoted as ${\bf u}$ which satisfies $${\bf Au} = \left[\begin{matrix} 0\\0\\ \vdots \\0\\1 \end{matrix}\right]$$
\end{frame}

\begin{frame}
Append a row of zeros to the bottom of ${\bf B^{-1}}$, and it can be seen that $$\left[\begin{matrix}\bf B_{(n-1)\times(n-1)} & \bf C_{(n-1)\times 1} \\\bf D_{1\times (n-1)} & \bf E_{1\times 1}\end{matrix}\right]\left[\begin{matrix}\bf B^{-1}_{(n-1)\times (n-1)} \\ \bf 0_{1\times (n-1)} \end{matrix}\right] = \left[\begin{matrix}\bf I_{(n-1)} \\ \bf error_{1\times (n-1)} \end{matrix}\right]$$

Then what should we do is add scaled $\bf u$ on the columns of the new matrix to eliminate the errors. At last append $\bf u$ to the right, then ${\bf A^{-1}}$ is calculated.
\end{frame}

\begin{frame}
	\section{Implementation}
	\frametitle{Python specifics included to improve coding efficiency}
	The basic object used is {\bf numpy.mat} to store the matrices. There are several reasons 

	\begin{itemize}
		\item It is more convenient for storing and generating test matrices using numpy commands
		\item Many loops and case statement in the implementation can be substituted by matrices operations (row operation, matrix partition, matrix multiplicaiton, etc.), and numpy.mat has adequette implementations of these operations.
	\end{itemize}
\end{frame}

\begin{frame}
	\section{References}
	\frametitle{References}
	\begin{itemize}
		\item Aravindh Krishnamoorthy, Deepak Menon (2013),  "Matrix Inversion Using Cholesky Decomposition"
		\item Trench, W. F. (1964). "An algorithm for the inversion of finite Toeplitz matrices." J. Soc. Indust. Appl. Math., v. 12, pp. 515–522.
		\item Musicus, B. R. (1988). "Levinson and Fast Choleski Algorithms for Toeplitz and Almost Toeplitz Matrices." RLE TR No. 538, MIT
	\end{itemize}
\end{frame}

\end{document}