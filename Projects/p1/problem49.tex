%
% do not add anything in this part
%
\ifx\onefile\undefined

	\documentclass{article}

	%if tcolorbox and tikz are installed use next line
	%\usepackage[tcbox]{projectalgo}
	\usepackage{projectalgo}

	% replace type by one of graph, math, combinatory, string, network, datastructure, ai, image
	\pbtype{type}

	\begin{document}
\fi

%
% things can be added below
%

\def\pbname{Matrix inversion} %change this, do not use any number, just the name

\section{\pbname} 

% only for overview, so short description (no more than 1-2 lines)
\begin{overview}
\item [Algorithm:] Cholesky Algorithm~(algo.~\ref{problem49a}), Levinson-Durbin Algorithm~(algo.~\ref{problem49b})
	% - replace nb with problem number (e.g. problem101)
	% -	must match the label of the algorithm 
	% - for more than one algo list each of them and use problem101a, problem101b, problem101c etc.
\item [Input:] A square matrix $A$
\item [Complexity:] $\mathcal{O}(n^3)$
\item [Data structure compatibility:] N/A
\item [Common applications:] Finding inverse, solving linear systems
\end{overview}



\begin{problem}{\pbname}
	Finding the inverse of a matrix
\end{problem}

\subsection*{Description}
Two matrices are named inverse to each other if their product is the identity matrix of their size. That is to say $$AA^{-1}=I_n$$

Finding the inverse of a matrix is very important in linear algebra and real life practices, so a good inversion method is necessary. The most intuitive method must be euqation solving, which means building a linear system with regard to vectors and solve it using elimination and substitution. But this can be time-consuming.

However, based on some special features of matrix, the time of calculating the inverse could be reduced. 

Cholesky algorithm is a inverse finding algorithm for Hermittian matrices. A $n\times n$ matrix $A$ is said to be Hermittian if it is conjugate symmetric. In another word $$a_{ij} = \bar{a_{ji}} \text{ for } 1\leq i,j \leq n$$

And if the matrix is a real matrix, being Hermittian is equivalent to being symmetric. 

Cholesky algorithm is based on the Cholesky decomposition of a Hermittian. Cholesky decomposition is decomposing a matrix into the product of an upper triangular matrix with its conjugate transpose. In a word, it is like $$A = R^*R$$ where $R$ is upper triangular.

After decomposing, the original equation can be divided into two equation systems . $$R^*RA^{-1}=I_n \Rightarrow \begin{cases}R^*y &= I_n\\RA^{-1}&=y\end{cases}$$

Solving this equation system is simpler since the coefficient matrices are now tiangular matrices and consequently methods like forward/backward substitution can be directly applied. What's more, based on the Hermittian property the amount of operations can be reduced by nearly half. The decomposition part requires $\frac{n^3}{6}$ operations and the back substitution is a $\mathcal{O}(n^2)$ process, so the complexity for Cholesky algorithm is $\mathcal{O}(n^3)$

Another algorithm is Levinson-Durbin algorithm. This algorithm is designed for a special type of matrices called symmetric Toeplitz matrices. Toeplitz matrices are suqare matrices who have same entries on every diagonals(not only main diagonal), which is to say $$A_{i,j}=a_{i-j} \text{ for } 1\leq i,j \leq n$$ where $\{a_k\}$ is a constant sequence whose index ranges from $-(n-1)$ to $(n-1)$

Levinson-Durbin algorithm uses the idea of recursion. Using recursive call, it first finds the inverse of the $(n-1)\times (n-1)$ matrix on the left-top of $A$(denoted as $A_{n-1}$), denoted as $A_{n-1}^{-1}$. Then it adds a row of $0$ at the the bottom of $A_{n-1}^{-1}$. 

It is clear that if you multiply the result of such operation with $A$, the result will only have differences with the first $(n-1)$ columns in $I_n$ on the last row.

The next step is to find the backward vector of $A$, which is a vector $b_n$ such that (here $e_n$ stands for the $n_{th}$ column in the identity matrix $I_n$) $$Ab_n = e_n$$

After finding this backward vector, weight it and add it to columns in $A_{n-1}^{-1}$ to eliminate the errors in the last row and the append it to the right of $A_{n-1}^{-1}$, you will get $A^{-1}$.

The process of finding $b_n$ is also recursive. And here another term called forward vector needs to be introduced, which is a vector $f_n$ which satisfies $$Af_n=e_1$$

Because the matrices studied are symmetric Toeplitz matrices, $f_n$ is just the reverse of $b_n$, which means $$f_n[i]=b_n[n-i+1] \text{ for } 1\leq i \leq n$$

Here is the process of finding $b_n$. First use recursive call to find $b_{n-1}$, and generate $f_{n-1}$ by reversing $b_{n-1}$. Append a 0 at the end of $f_{n-1}$ and the beginning of $b_{n-1}$, then the product of $A$ and the new two vectors(denoted as $f_{n}^{'},b_{n}^{'}$) will only have difference with $e_1$ and $e_n$ on the last/first row respectively. 

Denote the deviations as $\delta_n,\delta_1$, it can be seen that using linear combinations of these two vectors can eliminate the deviations and generate $b_n$ $$b_n = \frac{1}{1-\delta_1\delta_n}b_{n}^{'}-\frac{\delta_1}{1-\delta_1\delta_n}f_{n}^{'}$$

Levinson recursion originally serves to solve linear systems, and in that case it is of $\mathcal{O}(n^2)$ complexity. Because here the constant terms are vectors, the complexity will be $\mathcal{O}(n^3)$

% add comment in the pseudocode: \cmt{comment}
% define a function name: \SetKwFunction{shortname}{Name of the function}
% use the defined function: \shortname{$variables$}
% use the keyword ``function'': \Fn{function name}, e.g. \Fn{\shortname{$var$}}
\begin{Algorithm}[Cholesky Algorithm\label{problem49a}][H]
	% - replace nb with problem number (e.g. problem101)
	% -	must match the reference in the overview
	% - when writing more than one algo use problem101a, problem101b, problem101c etc.
	\SetKwFunction{chol}{Cholesky}	
	\Input{A Hermittian square matrix $A$}
	\Output{The inverse of $A$}
		\Fn{\chol{$A$}}{
			n $\leftarrow$ size of A\;
			R $\leftarrow$ size-n square matrix filled with zeros\;
			\For {i $\leftarrow$ 1 \KwTo n}{
				\For {j$\leftarrow$ i \KwTo n}{
					\For{k$\leftarrow$ 1 \KwTo i-1}{
						R[i,j] $\leftarrow R[i,j]+R[k,i]^*R[k,j]$\;
					}
					\uIf{i$\neq$j}{
						R[i,j] $\leftarrow \frac{A[i,j]-R[i,j]}{R[i,i]}$ \;
					}
					\Else{
						R[i,j] $\leftarrow \sqrt{A[i,j]-R[i,j]}$ \;
					}
				}
			}
			X $\leftarrow$ size-n square matrix filled with zeros\;
			\For {i $\leftarrow$ n \KwTo 1}{
				\For {j$\leftarrow$ i \KwTo n}{
					X[i,j] $\leftarrow \sum_{k=i+1}^{n}R[i,k]X[k,j]$\;
					\uIf{i=j}{
						$X[i,j] \leftarrow \frac{\frac{1}{R[i,j]}-X[i,j]}{R[i,j]}$\;
					}
					\Else {
						$X[i,j] \leftarrow \frac{0-X[i,j]}{R[i,j]}$\;
						$X[j,i] \leftarrow X[i,j]^*$\;
					} 
				}
			}
			\Ret{X}
		}
	\BlankLine

	\Ret{\chol{$A$}}

\end{Algorithm}

\begin{Algorithm}[Levinson-Durbin Algorithm\label{problem49b}][H]
	% - replace nb with problem number (e.g. problem101)
	% -	must match the reference in the overview
	% - when writing more than one algo use problem101a, problem101b, problem101c etc.
	\SetKwFunction{ld}{LevinsonDurbin}
	\SetKwFunction{bv}{BackwardVector}	
	\Input{A Symmetric Toeplitz matrix $A$}
	\Output{The inverse of $A$}
	\Fn{\ld{$A$}}{
		n $\leftarrow$ size of A\;
		\uIf{n=1}{
			\Ret{$[\frac{1}{A[1,1]}]$}
		}
		B $\leftarrow$ \ld{$A[1:n-1,1:n-1]$}\;
		Append a row of zeros to the bottom of $B$\;
		Append \bv{$A$} to the right of $B$\;
		\For {i $\leftarrow$ 1 \KwTo n-1}{
			d $\leftarrow \sum_{k=1}^{n}A[n,k]B[k,i]$\;
			\For {j $\leftarrow$ 1 \KwTo n}{
				$B[j,i] \leftarrow B[j,i]-d\times B[j,n]$\;
			}
		}
		\Ret{$B$}
	}
	\BlankLine
	\Fn{\bv{$A$}}{
		n $\leftarrow$ size of A\;
		\uIf{n=1}{
			\Ret{$[\frac{1}{A[1,1]}]$}
		}
		b $\leftarrow$ \bv{$A[1:n-1,1:n-1]$}\;
		Append a zero at the head of $b$\;
		f $\leftarrow$ b reversed\;
		$dn \leftarrow \sum_{k=1}^n A[n,k]f[k]$\;
		$d1 \leftarrow \sum_{k=1}^n A[1,k]b[k]$\;
		$b \leftarrow \frac{1}{1-d1\times dn}b-\frac{d1}{1-d1\times dn}f$ \;
		\Ret{$b$}
	}
	\BlankLine

	\Ret{\ld{$A$}}

\end{Algorithm}

\subsection*{References}
% list references where to find information on the given problem
% prefer books, research articles, or internet sources that are likely to remain available over time
% as much as possible offer several options, including at least one which provide a detailed study of the problem
% if available include links to programs/code solving the problem

\begin{itemize}\itemsep .125cm
	\item Aravindh Krishnamoorthy, Deepak Menon (2013),  "Matrix Inversion Using Cholesky Decomposition"
	\item Trench, W. F. (1964). "An algorithm for the inversion of finite Toeplitz matrices." J. Soc. Indust. Appl. Math., v. 12, pp. 515–522.
	\item Musicus, B. R. (1988). "Levinson and Fast Choleski Algorithms for Toeplitz and Almost Toeplitz Matrices." RLE TR No. 538, MIT
\end{itemize}

\ifx\onefile\undefined
	\end{document}
\fi
