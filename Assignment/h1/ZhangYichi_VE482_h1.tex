\documentclass{article}
\usepackage{bookmark}
\usepackage{color}
\usepackage{amsmath}
\usepackage{hyperref}
\usepackage{listings}
\usepackage{xcolor}
\usepackage{indentfirst}
\usepackage{graphicx}
\usepackage{amsfonts}
\usepackage{hyperref}
\usepackage[top=2cm, bottom=2cm, left=2cm, right=2cm]{geometry}  
\usepackage{algorithm}  
\usepackage{algorithmicx}  
\usepackage{algpseudocode} 
 
\renewcommand{\algorithmicrequire}{\textbf{Input:}}  
\renewcommand{\algorithmicensure}{\textbf{Output:}}  

\begin{document}
{\noindent {\bf Ex 1.} (1) First, to choose a k-key combination from n keys, there are $\binom{n}{k}$ different combinations}

For a certain k-key combination, the possibility that they lie into the slot is $\left(\frac{1}{n}\right)^k$. For all the other keys, the possibility that they do not go into that slot is $(1-\frac{1}{n})^{n-k}$

In conclusion, the possibility that $k$ keys go into a slot is $$(\frac{1}{n})^k(1-\frac{1}{n})^{n-k}\binom{n}{k}\\$$

{\noindent (2) From (1) we can see that for any integer between 1 and n, the possibility that the $i_th$ slot has k keys is $P_k$. However since there is possibility that more than one slot has $k$ keys, the possibility that there exists a slot which has $k$ keys is smaller than $nP_k$. And the possibility that that slot has most number of keys is even smaller so $$P_k^{'}<=P_k$$\\}

{\noindent (3) Because $\binom{n}{k}=\frac{n!}{k!(n-k)!}$}

According to stirling formula, we have $$\binom{n}{k}<\frac{en^{n+\frac12}e^{-n}}{2\pi k^{k+\frac12}(n-k)^{n-k+\frac12}e^{-n}}$$

Then $$P_k<\frac{e\sqrt{n}(n-1)^{n-k}}{2\pi k^{k+\frac{1}{2}}(n-k)^{n-k+\frac12}}<\frac{e\sqrt{n}}{2\pi k^{k+\frac12}\sqrt{n-k}}\times e^{k-1}<\frac{e^k}{k^k}\\$$

\hrule

\vskip 2em

{\noindent {\bf Ex 2.} The algorithm is given below}

\begin{algorithm}[H]  
    \caption{Finding Minimal Spanning Tree with Decreased Edge Weight}  
    \begin{algorithmic}[1] %每行显示行号  
        \Require Graph $G=<V,E>$, previous Minimal Spanning Tree $T$, the updated edge $e_0 = (m,n)$ and its new weight $w_e$  
        \Ensure A new Minimal Spanning Tree for $G$
            \State $E_p\gets Edges\ on\ the\ path \ from\ m\ to\ n\ in\ $
            \State $e_t \gets The\ edge\ in\ E_p\ with\ greatest\ weight$
            \If {$w_{e_t}>w_e$}
                \State $T\gets T\setminus e_t$
                \State $T\gets T+e$
            \EndIf
            \State \Return T
              
    \end{algorithmic}  
\end{algorithm}

\hrule
\vskip 2em

{\noindent {\bf Ex 3. }(2). (a) The algorithm is shown below.}


\begin{algorithm}[H]  
    \caption{Multiplication using recursion}  
    \begin{algorithmic}[1]  
        \Require Two numbers $x,y$ 
        \Ensure Their multiplication
            \Function {Multi} {$x,y$}
                \If {$x=0\ or\ y=0$}
                    \State \Return $0$
                \Else
                    \If {$y\equiv 0(mod\ 2)$}
                        \State \Return $Multi(x*2,y/2)$
                    \Else
                        \State \Return $Multi(x*2,(y-1)/2)+x$
                    \EndIf
                \EndIf
            \EndFunction
            \State \Return $Multi(x,y)$
              
    \end{algorithmic}  
\end{algorithm}

{\noindent (b) Consider if $y$ is even then $xy = (2x)\times \frac{y}{2}$}

If $y$ is odd then $xy = x(1+2\times\frac{y-1}{2})=x+(2x)\times\frac{y-1}{2}$

So it can be seen that the recurrance relation is correct. Then because the parameter $y$ is divided by 2 repeatedly, it will finally com to 0, so the boundary condition is reachable.

So the algorithm is correct.\\

\hrule 
\vskip 2em

{\noindent {\bf Ex 4. } 7 races should be applied to those horses. The algorithm is shown below.}

First, devide the 25 horses into 5 groups of five horses, and perform a race for each group.

Then pick all the 5 No. 1 horses and let them do the race. Consider in this race $a_i$ represents the $i_th$ fastest horse, and obviously $a_1$ is the No.1 horse in all 25 horses.  

Then perform the last race for the No.2 and No.3 horse in $a_1$'s original group, $a_2$ and the No.2 horse in his group and $a_3$. The fasetest and second fastest horse will be the No.2 and No.3 horse for all 25.\\

\hrule 
\vskip 2em

{\noindent {\bf Ex 5. } (1) The two algorithms can both solve the problem in some situation, but cannot for some others. }

Consider the following two situations:

1. $S=\{2,4,5\},n=5$

2. $S=\{2,4,5\},n=6$

It is obvious that the first algorithm can solve the second situation but not the first, while the situation for the second algorithm is just the opposite.

So in conclusion, the two both work in some situation but both have their limits.\\

{\noindent (3). The problem is the {\bf coin change problem}. }

{\bf Description: } Giving a method of changing \$n money with $\$a_1,\$a_2,...,\$a_n$ coins, which uses the smallest number of coins.  

{\bf Greedy Algorithm: }Always use the coin with the largest value to change, until the remaining number is smaller than that value. Then uses the second largest value coins and repeat the whole process.

{\bf Counter Example: } When $n=15$, $a=\{1,5,6,7\}$, The greedy algorithm will give the method that uses $1\times \$7,1\times \$6$ and $2\times \$1$, which costs 4 coins. But in fact the globally optimal answer is using three \$5 coins.
\end{document}