\documentclass{article}
\usepackage{bookmark}
\usepackage{color}
\usepackage{amsmath}
\usepackage{hyperref}
\usepackage{listings}
\usepackage{xcolor}
\usepackage{indentfirst}
\usepackage{graphicx}
\usepackage{amsfonts}
\usepackage{hyperref}
\usepackage[top=2cm, bottom=2cm, left=2cm, right=2cm]{geometry}  
\usepackage{algorithm}  
\usepackage{algorithmicx}  
\usepackage{algpseudocode}
\pagestyle{headings}
\markright{\large Zhang Yichi 516370910260\hfill VE477 h4\hfill}
\renewcommand{\algorithmicrequire}{\textbf{Input:}}  
\renewcommand{\algorithmicensure}{\textbf{Output:}}  

\begin{document}
{\noindent {\bf Ex 1.} 1. For $2^{64}$ operations, it needs about $\frac{2^{64}}{2^{50}} = 2^{14}$s, and for $2^{80}$ operations, $2^{30}$s is needed\\

{\noindent 2. For $2^{64}$ operation, it needs $\frac{2^{64}}{24*3600*3.8*2^{30}}\approx 52327$ desktops. And for $2^{80}$ operations for a month, it needs, about $114309202$ desktops.\\}

{\noindent 3. For $2^{64}$ bits, it needs $\frac{2^{64}}{2^44}=2^{20}$ hardrives, and for $2^{80}$ it needs about $2^{36}$ hardrives.\\}

\hrule

\vskip 2em

{\noindent {\bf Ex 2.} It can be done in the following way }\\

Let us use an array $R$ to store the subset. For the first $k$ subsets, directly put them into $R$. For the items afterwards, suppose for number $m$ element $(k<m<n)$, generate a random number $i$ ranging from $1$ to $m$. If the number is in the range $[1,k]$, then replace $R[i]$ by the $m_{th}$ element, otherwise do nothing. When all items are checked, the subset is generated.\\ 

\hrule
\vskip 2em

{\noindent {\bf Ex 3. }1.  Pseudo code can be shown below

\begin{algorithm}[H]  
    \caption{Triangle Layer Sum}  
    \begin{algorithmic}[1]  
        \Require Number of Layer $i$
        \Ensure The sum of elements on the $i_{th}$ layer
            \Function {ThreePower}{i}
                \If {$i=1$}
                    \State \Return 1
                \EndIf
                \Return $3\times ThreePower(i-1)$
            \EndFunction
            \State \Return $ThreePower(i)$            
    \end{algorithmic}  
\end{algorithm}

{\noindent 2. It is a $O(n)$ algorithm and for its correctness.}

According to the rule, we can see that to generate the elements in layer $i+1$, the elements in layer $i$ are used 3 times except the two $1$s in both sides --- They are used 2 times. And because there are two new $1$s in the layer, the recurrance relation can be written as $$f_n = 3(f_{n-1}-2)+2\times 2+2 = 3f_{n-1}$$

And because $f_1 = 1$, then it can be seen that $f_n = 3^{n-1}$\\

\hrule 
\vskip 2em

{\noindent {\bf Ex 5. } 2. Given a subset of points, it uses at worst $O(n^2)$ time to check whether the points are adjacent to each other by brute force, so an answer can be checked in polynomial time, then it is an $\mathcal{NP}$ problem\\}

{\noindent 3. Let G be a graph whose vertices are $(v,c)$, where $c$ is one of clauses and $v$ is a literal appears in that clause (with or without negation), and denote $(v_1,c_1)$, $(v_2,c_2)$ as connected if $c_1\neq c_2$ and $u$ is not $\neg v$. Then the existance of k-clique in the graph can be determined by whether $\mathcal{F}$ is satisfied\\}

{\noindent 4. The problem is $\mathcal{NP}$-complete\\}


\hrule 
\vskip 2em

{\noindent {\bf Ex 6. } 2. It is the problem to determine whether a undirected graph has a independent subset of size $k$}\\

{\noindent 3. Given a subset of points, it uses at worst $O(n^2)$ time to check whether the points are not adjacent to each other by brute force, so an answer can be checked in polynomial time, then it is an $\mathcal{NP}$ problem\\}

{\noindent 4. Just using the complementary graph of $G$ can reduce the problem to clique problem}

{\noindent 5. It is also $\mathcal{NP}$-complete}


\end{document}