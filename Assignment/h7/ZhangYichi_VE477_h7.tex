\documentclass{article}
\usepackage{bookmark}
\usepackage{color}
\usepackage{amsmath}
\usepackage{hyperref}
\usepackage{listings}
\usepackage{xcolor}
\usepackage{indentfirst}
\usepackage{graphicx}
\usepackage{amsfonts}
\usepackage{hyperref}
\usepackage[top=2cm, bottom=2cm, left=2cm, right=2cm]{geometry}  
\usepackage{algorithm}  
\usepackage{algorithmicx}  
\usepackage{algpseudocode}
\pagestyle{headings}
\markright{\large Zhang Yichi 516370910260\hfill VE477 h6\hfill}
\renewcommand{\algorithmicrequire}{\textbf{Input:}}  
\renewcommand{\algorithmicensure}{\textbf{Output:}}  

\begin{document}
{\noindent {\bf Ex 1.} 1. The least probability occurs when $t=2$, and in that case the least probability is $\binom{2}{2}/\binom{6}{2}=1/15$\\ 


{\noindent 3. It is obvious that $z_0 = 4/p_0-1=59$, and for the recurrance relation\\}

\begin{align*}
    p_{k+1}&=p_k-\frac{p_k^2}{4}\\
    \Rightarrow \frac{4}{z_{k+1}+1}&= \frac{4}{z_k+1}-\frac{(\frac{4}{z_k+1})^2}{4}\\
    \Rightarrow 4(z_k+1)^2 &= 4(z_k+1)(z_{k+1}+1)-4(z_{k+1}+1)\\
    \Rightarrow z_{k+1} &= \frac{(z_k+1)^2-z_k}{z_k}\\
    &=z_k+\frac{1}{z_k}+1
\end{align*}

\hrule

\vskip 2em

{\noindent {\bf Ex 2.} 1.In matrices view, the thing can be done by performing row operation on the coefficient matrix. The system can be written as $$\left[\begin{matrix}3&1&2&0&0&0\\1&1&3&1&0&0\\2&2&5&0&1&0\\4&1&2&0&0&1\end{matrix}\right]x = \left[\begin{matrix}z\\30\\24\\36\end{matrix}\right]$$}


Do row operation to eliminate the elements in first column using the last row, the matrix will be $$\left[\begin{matrix}0&1/4&1/2&0&0&-3/4\\0&3/4&3/2&1&0&-1/4\\0&3/2&4&0&1&-1/2\\4&1&2&0&0&1\end{matrix}\right]x = \left[\begin{matrix}z-27\\21\\6\\36\end{matrix}\right]$$

Next is to eliminate the elements in second column for the first two rows using the third row, then it will be $$\left[\begin{matrix}0&0&-1/6&0&-1/6&-2/3\\0&0&-1/2&1&-1/2&0\\0&3/2&4&0&1&-1/2\\4&1&2&0&0&1\end{matrix}\right]x = \left[\begin{matrix}z-28\\18\\6\\36\end{matrix}\right]$$

Now all coefficients are nonpositive in the first row, and then it can be seen that the max value of $z$ should be $28$\\

\hrule
\vskip 2em

{\noindent {\bf Ex 3. }It is possible, and it can be implemented using unsorted array for stacks and use another stack to store the "history record" of minimum values. The top element for the second stack is exactly the minimum value\\}

The two stacks are initialized as empty. In pushing, if the stack is originally empty or the pushed value is smaller than or equal to the minimum value, push it also into the second stack. And in popping, if a minimum is popped, also pop it from the second stack. Then in that case, all operations are in constant time.\\

\hrule
\vskip 2em

{\noindent {\bf Ex 4. } This can be proven using contradiction. Suppose that both are solvable. Then it can be deduced that $$V^Tx = (M^Ty)^Tx=y^TMx>0$$}

The fact that $V^Tx>0$ implies that $V$ and $x$ are not zero vectors, and $M^Ty=V$ implies that $y$ isn't neither. Because $Mx\neq 0$, then it means that $y^TMx\neq 0$, which contradicts to the previous conclusion, so the two cannot be solved together.
\end{document}